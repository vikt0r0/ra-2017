\documentclass[12pt]{article}

% This first part of the file is called the PREAMBLE. It includes
% customizations and command definitions. The preamble is everything
% between \documentclass and \begin{document}.

\usepackage[margin=1in]{geometry}  % set the margins to 1in on all sides
\usepackage{graphicx}              % to include figures
\usepackage{amsmath,bm}            % great math stuff
\usepackage{amsfonts}              % for blackboard bold, etc
\usepackage{amsthm}                % better theorem environments
\usepackage{listings}
\usepackage{hyperref}
\usepackage{tikz}
\usepackage[]{algorithm2e}
\usepackage{parskip} 			   % no paragraph indentation

\usetikzlibrary{arrows,automata}


% various theorems, numbered by section

\newtheorem{thm}{Theorem}[section]
\newtheorem{lem}[thm]{Lemma}
\newtheorem{prop}[thm]{Proposition}
\newtheorem{cor}[thm]{Corollary}
\newtheorem{conj}[thm]{Conjecture}

\DeclareMathOperator{\id}{id}

\newcommand{\bd}[1]{\mathbf{#1}}  % for bolding symbols
\newcommand{\RR}{\mathbb{R}}      % for Real numbers
\newcommand{\ZZ}{\mathbb{Z}}      % for Integers
\newcommand{\col}[1]{\left[\begin{matrix} #1 \end{matrix} \right]}
\newcommand{\comb}[2]{\binom{#1^2 + #2^2}{#1+#2}}

\lstset{ % General setup for the package
    language={[LaTeX]TeX},
    basicstyle=\footnotesize\sffamily,
    tabsize=4,
    columns=fixed,
    keepspaces,
    commentstyle=\color{red},
    keywordstyle=\color{blue},
    xleftmargin=.1\textwidth,
    xrightmargin=.1\textwidth
}

\begin{document}

\nocite{*}


\title{Randomized Algorithms \\
       Assignment 1}

\author{Viktor Hansen \& Simon Rueskov Schleicher}

\maketitle

\begin{abstract}
  This is the first weekly assignment for the Randomized Algorithms course offered at The Department of Computer Science, Uni. Copenhagen.
\end{abstract}

\pagebreak

\section*{Exercise 1.2}
Consider the graph $G=(V,E)$ consisting of two $(\left|V\right|/2)$-cliques, $L$ and $R$, connected by a single edge (assume that $|V|$ is even, without loss of generality). The min-cut of $G$ has size 1. If at any point, the algorithm coalesces any pair of vertices that are not both from $L$ or $R$, respectively, the size of the min-cut increases in the resulting graph, and it can not be found in subsequent iterations. Let $\varepsilon$ denote the event that a pair of vertices from either clique are chosen. Then 
\begin{align*}
\mathbf{Pr}\left[ \varepsilon \right] = \frac{\binom{|L|}{2} + \binom{|R|}{2}}{\binom{|L|+|R|}{2}}
\end{align*}

\section*{Exercise 1.3}
To obtain a Las Vegas algorithm from $A$, we run $A$ on $\Pi$ and verify the solution in time $T(n) + t(n)$. We repeat this process until the correct answer is given by $A$. Let $N$ be the random variable denoting the number of times $A$ must be run on $\Pi$ to obtain the correct answer. Then the expected running time of the Las Vegas algorithm is given by $\mathbf{E}\left[ \left(T(n) + t(n)\right) \cdot N \right] = \left(T(n) + t(n)\right) \cdot \mathbf{E} \left[ N \right] = \left(T(n) + t(n)\right) \cdot 1/\gamma{(n)}$ since $N$ follows a geometric distribution with parameter $\gamma{(n)}$.

\section*{Problem 1.1}
\paragraph{a)} For $k \geq 1$, let $s \in \left\{ H,T \right\}^k$ denote a $k$-coin toss sequence, s.t., for instance, $HT$ is a 2-coin toss sequence in which the first toss is a heads and the second is a tails. Then $HT$ and $TH$ can be viewed as the events heads and tails for a fair coin, respectively, since they are equiprobable, i.e. $\textbf{Pr}\left[ HT \right] = \textbf{Pr}\left[ TH \right] = p(1-p)=(1-p)p$. Let $N$ be the random variable denoting the number of independent 2-coin sequences needed to get either $HT$ or $TH$. Then $N$ follows a geometric distribution with parameter $p(1-p) + (1-p)p = 2p(1-p)$ since the events are independent, and $\mathbf{E}\left[ N \right] = 1/2p(1-p)$ which is no more than $1/p(1-p)$.

\paragraph{b)} Assume (without loss of generality) that $k$ is even. Given a sequence $s \in \left\{ H,T \right\}^k$, an extension of the above scheme does the following: 1) it scans through $s$ two elements at a time and extracts tosses in the manner described in a). 2) it subsequently removes the pairs corresponding to those tosses from the $s$. It is easily verified that the resulting list has the form $\left\{ HH,TT \right\}^{k'}$ where $k' \leq k$. 3) it removes every other member of $s$, once again producing a sequence of the form $\left\{ H,T \right\}^k$. Steps 1), 2) and 3) are repeated until the list is empty, and the extracted tosses are reported.

\section*{Problem 1.4}
\paragraph{a)} One solution generates a list of all $n!$ possible permutations and subsequently generates a random index into the list. Generating one such index requires $\lceil \log_2(n!) \rceil$ random bits and can be generated s.t. it falls within bounds of the list with probability $n! / 2^{\lceil \log_2(n!) \rceil} = \log_2(n!) / \lceil \log_2(n!) \rceil \geq \log_2(n!) / (\log_2(n!)+1) \geq 1/2$. Let $N$ be the random variable denoting the number of indices that need to be generated to make the index falls within the bounds of the list. Then the expected total number of bits needed to generate such an index is given by $\textbf{E}\left[ N \cdot \lceil \log_2(n!) \rceil \right] = \textbf{E}\left[ N \right] \cdot \lceil \log_2(n!) \rceil \leq 2 \cdot \lceil \log_2(n!) \rceil \leq 2\log_2(n!) + 1 \leq (n+1) \log_2 \left( \frac{n+1}{e} \right) + 2 = O(n \log n)$ since $N$ is geometrically distributed with parameter $\geq 1/2$. For the lower bound, we get $\textbf{E}\left[ N \right] \cdot \lceil \log_2(n!) \rceil \geq 1 \cdot \lceil \log_2(n!) \rceil = \Theta(n \log_2 n)$ by a similar reasoning. The time and space complexities are an impractical $\Theta(n!)$.

Another approach creates a bijection $f : [1 \hdots n] \rightarrow [1 \hdots n]$ by randomly assigning $\lceil \log_2 n \rceil$ bits to each element in the domain of $f$ corresponding to the elements of the range. Let $\varepsilon_i$ denote the event that the $i$th element of the domain is assigned a bitstring corresponding to a valid element in the range i.e. s.t. $f$ is a bijection.  There are $n-i+1$ valid indices and thus
\begin{align*}
\mathbf{Pr}\left[ \varepsilon_i \right] = \frac{n-i+1}{2^{\lceil \log_2 n \rceil}} \geq \frac{1}{2^{\lceil \log_2 n \rceil}} \geq \frac{1}{2^{\log_2 (n) + 1}} = \frac{1}{2n}
\end{align*}
Let $N_i$ be the random variable denoting the number of bitstrings required to generate a valid element of the range for the $i$th element in the domain. Then, the total expected number of bits required is 
\begin{align*}
\mathbf{E} \left[ \sum_{i=1}^{n} N_i \cdot \lceil \log_2 n \rceil \right] = \sum_{i=1}^{n} \mathbf{E}\left[ N_i \right] \cdot \lceil \log_2 n \rceil \leq \sum_{i=1}^{n} 2n \cdot \lceil \log_2 n \rceil = O(n^2 \log_2 n)
\end{align*}
For the lower bound, we note that we need at least one try for each element in the domain and get
\begin{align*}
\sum_{i=1}^{n} \mathbf{E}\left[ N_i \right] \cdot \lceil \log_2 n \rceil \geq \sum_{i=1}^{n} 1 \cdot \lceil \log_2 n \rceil = \Omega(n \log_2 n)
\end{align*}
For the running time, the algorithm checks whether a generated element is already used in the permutation in $\Theta(n)$ time, yielding expected running times of $\Omega(n^2)$ and $O(n^3)$.

\paragraph{b)} The claim is correct, since there is a 0 probability of collisions occurring if the numbers are picked from the interval, uniformly, at random. However an unbounded number of bits are required to represent such a number, and hence a lazy approach, such as the one described in c) is required.

\paragraph{c)} 


\section*{Problem 1.8}


\end{document}
