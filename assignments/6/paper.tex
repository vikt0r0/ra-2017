\documentclass[12pt]{article}

% This first part of the file is called the PREAMBLE. It includes
% customizations and command definitions. The preamble is everything
% between \documentclass and \begin{document}.

\usepackage[margin=1in]{geometry}  % set the margins to 1in on all sides
\usepackage{graphicx}              % to include figures
\usepackage{amsmath,bm}            % great math stuff
\usepackage{amsfonts}              % for blackboard bold, etc
\usepackage{amsthm}                % better theorem environments
\usepackage{listings}
\usepackage{hyperref}
\usepackage{tikz}
\usepackage[]{algorithm2e}
\usepackage{parskip} 			   % no paragraph indentation

\usetikzlibrary{arrows,automata}


% various theorems, numbered by section

\newtheorem{thm}{Theorem}[section]
\newtheorem{lem}[thm]{Lemma}
\newtheorem{alg}[thm]{Algorithm}
\newtheorem{dfn}[thm]{Definition}
\newtheorem{prop}[thm]{Proposition}
\newtheorem{cor}[thm]{Corollary}
\newtheorem{conj}[thm]{Conjecture}

\DeclareMathOperator{\id}{id}

\newcommand{\bd}[1]{\mathbf{#1}}  % for bolding symbols
\newcommand{\RR}{\mathbb{R}}      % for Real numbers
\newcommand{\ZZ}{\mathbb{Z}}      % for Integers
\newcommand{\col}[1]{\left[\begin{matrix} #1 \end{matrix} \right]}
\newcommand{\comb}[2]{\binom{#1^2 + #2^2}{#1+#2}}

\lstset{ % General setup for the package
    language={[LaTeX]TeX},
    basicstyle=\footnotesize\sffamily,
    tabsize=4,
    columns=fixed,
    keepspaces,
    commentstyle=\color{red},
    keywordstyle=\color{blue},
    xleftmargin=.1\textwidth,
    xrightmargin=.1\textwidth
}

\begin{document}

\nocite{*}


\title{Randomized Algorithms \\
       Assignment 6}

\author{Viktor Hansen \& Simon Rueskov Schleicher}

\maketitle

\begin{abstract}
  This is the sixth weekly assignment for the Randomized Algorithms course offered at The Department of Computer Science, Uni. Copenhagen.
\end{abstract}

\pagebreak


\section*{Exercise 7.4}
If we use the fingerprint function used for polynomial identities as F, we can solve the problem in the same way, but this takes more time.\\
With the fingerprint function described in the chapter, when we have calculated $F(X(j))$ we can calculate $F(X(j+1))$ as $F(X(j+1))=(F(X(j))-x_j*2^m)*2+x_{m+1}\text{ mod }p$, which can be done in $O(1)$ time. The reason we can do this is that if we call the m-bit integer starting at $x_j$ for $Z_j$ and the m-bit integer starting at $x_{j+1}$ for $Z_{j+1}$, then to change $Z_j$ into $Z_{j+1}$ we need to remove the first bit (by subtracting $x_j*2^m$), shift everything 1 place to the left (by multiplying with $2$) and add $x_{m+1}$. And calculating this modulo p doesn't change the result.\\
With the fingerprint function used for polynomial identities on the other hand, all of the bits are multiplied by something different whether we're calculating $F(X(j))$ or $F(X(j+1))$, and there's no easy conversion between this. So we have to calculate for all m bits each time thus taking $O(m)$ time to get from $F(X(j))$ to $F(X(j+1))$.

\section*{Problem 7.1}
Define the multivariate polynomial $Q(x_1,...,x_n)=ABr-Cr$ where r is the vector consisting of $x_1,...,x_n$ in that order. We have $AB=C\Rightarrow ABr=Cr\Rightarrow ABr-Cr=0$ for any $r$, so we get $Q\equiv0$ if $AB=C$. If $AB\neq C$ we get $AB-C\neq0$, so there exists an $r$, where $ABr-Cr\neq0$. We therefore get $Q\equiv 0$ if and only if $AB=C$.\\
We choose the set $S=\{0,1\}$, and get $|S|=2$. According to Theorem 7.2 (Schwartz-Zippel Theorem), if we let $r_1,...,r_n$ be chosen independently and uniformly at random from $S$, then $Pr[Q(r_1,...,r_n)=0|Q(x_1,...,x_n)\not\equiv0]\leq\frac{d}{|S|}=\frac{d}{2}$. With $d=1$ we get $Pr[Q(r_1,...,r_n)=0|Q(x_1,...,x_n)\not\equiv0]\leq\frac{1}{2}$.\\
$AB\neq C$ is equivalent to $Q(x_1,...,x_n)\not\equiv0$. So if we assume $AB\neq C$, we get $Pr[Q(r_1,...,r_n)=0]\leq\frac{1}{2}$. From our definition of $Q$ we get $Q(r_1,...,r_n)=0\Leftrightarrow ABr-Cr=0\Leftrightarrow ABr=Cr$, so we get $Pr[Q(r_1,...,r_n)=0]\leq\frac{1}{2}\Leftrightarrow Pr[ABr=Cr]\leq\frac{1}{2}$, which is what we wanted to show.

\section*{Problem 7.4}
We call the integers in $S_1$ $s_1,...,s_n$ and the integers in $S_2$ $s_{n+1},...,s_{2n}$. We say that the lowest value of the integers in $S_1$ and $S_2$ is $m_1$ and the highest value of the integers in $S_1$ and $S_2$ is $m_2$. We define $m=m_2-m_1$. We can find $m_1$ and $m_2$ in $O(n)$ time.\\
A way to solve the problem involving a verification of a polynomial identity, is by defining $Q(x_{m_1},...,x_{m_2})=\Sigma_{i=1}^ns_ix_{s_i}-\Sigma_{i=n+1}^{2n}s_ix_{s_i}$. A calculation of $Q$ takes $O(n+m)$ time and requires $O(m)$ space.\\
We need to show that if $S_1$ and $S_2$ are identical in the sense that each integer occurs the same number of times in both sets, then $Q\equiv0$. We can sort each of the sets individually without changing the result of $Q$, and then we see $\Sigma_{i=1}^ns_ix_{s_i}-\Sigma_{i=n+1}^{2n}s_ix_{s_i}=\Sigma_{i=0}^ns_ix_{s_i}-s_{i+n}x_{s_{i+n}}=\Sigma_{i=0}^ns_ix_{s_i}-s_ix_{s_i}=0$. We also need to show that if $S_1$ and $S_2$ are not identical in the sense that each integer occurs the same number of times in both sets, then $Q\not\equiv0$. In this case there has to be an integer that occurs more times in $S_1$ than in $S_2$. Call this integer j. We get $Q(0,...,x_j=1,...,0)>0$, so $Q\not\equiv0$.\\
Now we have $Q\equiv0$ if and only if $S_1$ and $S_2$ are identical in the sense that each integer occurs the same number of times in both sets, and we can use Theorem 7.2 to solve the problem.\\
Comparing this solution with the deterministic solution given in the book, we see that if the integers in the sets has a small range ($m$ is small), then this solution is faster. But for large ranges of integers ($m$ is large) this solution is both slower and uses a lot of space.

\section*{Summary}
If we have 3 matrices $A$, $B$ and $C$, and we want to verify that $AB=C$, then we can use Theorem 7.1:\\
\textbf{Theorem 7.1: } Let $A$, $B$ and $C$ be $n*n$ matrices over $F$ such that $AB\neq C$. Then for r chosen uniformly at random from $\{0,1\}^n$, $Pr[ABr=Cr]\leq\frac{1}{2}$.\\
We of course also have that if $AB=C$, then $ABr=Cr$.\\
\textbf{Theorem 7.2 (Schwartz-Zippel Theorem): } Let $Q(x_1,...,x_n)\in F[x_1,...,x_n]$ be a multivariate polynomial of total degree d. Fix any finite set $S\subseteq F$, and let $r_1,...,r_n$ be chosen independently and uniformly at random from $S$. Then $$Pr[Q(r_1,...,r_n)=0|Q(x_1,...,x_n)\not\equiv0]\leq\frac{d}{|S|}.$$
Perfect Matchings in Graphs:\\
\textbf{Theorem 7.3 (Edmonds' Theorem): } Let $A$ be the $n*n$ matrix obtained from $G(U,V,E)$ as follows: $A_{ij}=x_{ij}$ if $(u_i,v_j)\in E$ and $A_{ij}=0$ otherwise. Define the multivariate polynomial $Q(x_{11},x_{12},...,x_{nn})$ as being equal to det($A$). Then, G has a perfect matching if and only if $Q\not\equiv0$.\\
Verifying Equality of Strings:\\
Interpret the data as $n$-bit integers $a$ and $b$, by defining $a=\Sigma_{i=1}^na_i2^{i-1}$ and $b=\Sigma_{i=1}^nb_i2^{i-1}$. Define the fingerprint function $F_p(x)=x\text{ mod }p$ for a prime p. For any number k, let $\pi(k)$ be the number of distinct primes less than k. Choose a $\tau=tn\log tn$ for large t. Then we have the following Theorem:\\
\textbf{Theorem 7.5: } $Pr[F_p(a)=F_p(b)|a\neq b]\leq\frac{n}{\pi(\tau)}=O(\frac{1}{t})$.\\
Pattern matching:\\
A text is a string $X=x_1x_2...x_n$ and a pattern is a string $Y=y_1y_2...y_m$, both over a fixed finite alphabet $\Sigma$, such that $m\leq n$. The pattern matching problem is to find out if m occurs anywhere in n. The simple brute force deterministic algorithm takes $O(nm)$ time. We can make a Monte Carlo algorithm to solve this problem:\\
For any string $Z\in\{0,1\}^m$, interpret $Z$ as an $m$-bit integer and define $F_p(Z)=Z\text{ mod }p$. Assume that $p$ is chosen uniformly at random from the set of primes smaller than a threshold $\tau$. We get $Pr[F_p(Y)=F_p(X(j))|Y\neq X(j)]\leq\frac{m}{\pi(\tau)}=O\left(\frac{m\log\tau}{\tau}\right).$. $Pr[\text{a false match occurs}]=O(\frac{1}{n})$.\\
\textbf{Theorem 7.6: } The Monte Carlo algorithm for pattern matching requires $O(n+m)$ time and has a probability of error $O(1/n)$.

\end{document}
